\section{Introduction}
In recent years, the attacks based on malicious scripts (VBA) from Microsoft Office documents has been growing in popularity and so has the necessity of detecting them. While originally the VBA language was developed as a powerful scripting language to help users automate tasks and create macro-driven applications, it became a prevalent infection vector due to its extended capabilities, such as accessing the native Windows system-calls. As a response to the growing threat of malicious macros, Microsoft has implemented several security features to prevent the execution of unwanted code. Nowadays, as a response to these measures, the attackers distribute the malicious documents through spear-phishing emails and use social engineering techniques in order to persuade the victim to enable the execution of the malicious macros.
\par
This paper aims to explain the current state of detections regarding these types of malicious files and analyzes the possibility of improving existing macro detections via machine learning techniques. We will consider a lightweight approach due to the limitations of being used in anti-malware solution. Our solution is based on a derived version of the One Side Classifier algorithm which builds a detection model focused on the properties of the macro code extracted from the VBA project of Microsoft Office files. The paper also explores the advantages of using features that are not based around a particularity of the malicious file.
