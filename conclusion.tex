\section{Conclusion}
\par
If we evaluate the current threat landscape, we observe that macro malware has become a prevalent security problem. 
Traditional “signature-based” detections have the advantage of being a safe detection method with low false positive rates, being easy to generate and maintain. Although effective on short-term, these methods do not produce a solution which has the capacity to generalise the malicious features of new threats and can be easy to bypass. Analyzing the results presented in this paper, we can state that our solution, along with existing protection measures is a good step toward a highly generic and accurate detection system. 
\par
We acknowledge the fact that our perceptron model is more difficult to maintain and train and requires a large collection of samples and compute power. Despite these requirements, this type of model is not as volatile and is capable of better generalising the characteristics of different samples.
\par
In a real world scenario, we consider that a middle way between these solutions should be chosen. The traditional methods provide stability to the detection process of known malware families, while the perceptron model can be more exploratory in nature and discover new unknown threats.
\par
Consequently, the solution presented in this paper is essential to a cutting edge security solution and an actual necessity in the current state of growing macro malware threats.
