\section{Problem description}

The key points of the VBA-based malicious software are the accessibility and the popularity of Office Products. This is due to the fact that those products offer a scripting component named \textit{macro}. A \textit{macro} is a scripting system initially aimed at automating certain tasks. Being written in Visual Basic, it has access to the majority of the functions implemented in Windows API. This, along with the mass usage of Office products, both in business and non-business fields, makes it one of the favourite tools for malware creators. According to a statistic made by Spiceworks\footnote{https://community.spiceworks.com/software/articles/2873-data-snapshot-the-state-of-productivity-suites-in-the-workplace}, nearly 82\% of companies use programs from the Microsoft Office package. Knowing this, it is easier for an attacker to build a malicious program which will be granted to run, with a certain probability, on the target machine. However, over the past years, Microsoft implemented different security patches which fixed known vulnerabilities and nowadays the execution of a macro without user consent is denied. Because of these measures, the attackers had to implement social engineering methods to trick users into executing malicious macros.
\par
Malicious macros are used both as a delivery method and as a standalone malicious entity. If, in the past, the main target of a malicious document was to spread itself across the network, nowadays the potential offered by those products made the malware creators to rethink their usage. It is more common to see a document whose target is to steal sensitive information from companies or to deliver a destructive component, for example, a ransomware. According to CISCO\footnote{https://www.cisco.com/c/dam/m/digital/elq-cmcglobal/witb/acr2018/acr2018final.pdf}, more than 38\% of malicious file extensions found in 2017 in groups of malicious prevalent files were Microsoft Office formats.
\par
Considering the above observation, we developed a solution to detect malicious macros inside an office file and keep the rate of false positives as low as possible, all with respect to performance requirements. Our system is based on a derived version of the perceptron, OSC\cite{OSC}, and it is used over a set of features extracted from the binary Visual Basic Application, which is found inside an office document. The system extracts the macros within a file, applies a lexical processing, extracts the sequence of features and feeds them to the machine learning algorithm. In this way, we provide a complete system to detect if an Office file is indeed malicious or not.
\par
Moreover, taking into account that the OSC is keeping the rate of false positives for its models at a small percentage, we can easily use and integrate them into an AV solution. Having a small amount of false positives is mandatory due to the fact that Office products are used in critical business infrastructures.
