%\documentclass[twoside]{article}
\documentclass[10pt, conference, compsocconf]{IEEEtran}
%
\usepackage{makeidx}  % allows for indexgeneration
\usepackage{longtable}
\usepackage{url}
\usepackage{algorithm}
%\usepackage{algorithmicx}
\usepackage{algpseudocode}
\usepackage{color}
\usepackage{graphicx}
\usepackage{amsmath}
\usepackage{amsfonts}
\usepackage{pgfplots}
\usepackage{pgfplotstable}

\usepackage{mathtools}

\DeclarePairedDelimiter\abs{\lvert}{\rvert}%
\DeclarePairedDelimiter\norm{\lVert}{\rVert}%

% Swap the definition of \abs* and \norm*, so that \abs
% and \norm resizes the size of the brackets, and the 
% starred version does not.
\makeatletter
\let\oldabs\abs
\def\abs{\@ifstar{\oldabs}{\oldabs*}}
%
\let\oldnorm\norm
\def\norm{\@ifstar{\oldnorm}{\oldnorm*}}
\makeatother

\usepackage[hidelinks]{hyperref}
\hypersetup{colorlinks=true,linktoc=all,allcolors=blue}

\usepackage{array}
\usepackage{longtable}

% Metadata Information


% Package to generate and customize Algorithm as per ACM style
%\usepackage[ruled]{algorithm2e}
%\SetAlFnt{\algofont}
%\SetAlCapFnt{\algofont}
%\SetAlCapNameFnt{\algofont}
%\SetAlCapHSkip{0pt}
%\IncMargin{-\parindent}
%\renewcommand{\algorithmcfname}{ALGORITHM}

\begin{document}
	
\title{Improving Detection of Malicious Office Documents \\ using \\ One-Side Classifiers}
\author{

	\IEEEauthorblockN{Silviu - Constantin Vi\c tel\\}
	\IEEEauthorblockA{"Al.I. Cuza" University - Faculty of Computer Science\\
	Bitdefender Cyber Threat Intelligence Lab\\
	Ia\c si, Rom\^ ania\\
	Email: svitel@bitdefender.com\\} 

	\and 
	\IEEEauthorblockN{Gheorghe Balan\\}
	\IEEEauthorblockA{"Al.I. Cuza" University - Faculty of Computer Science\\
	Bitdefender Cyber Threat Intelligence Lab\\
	Ia\c si, Rom\^ ania\\
	Email: gbalan@bitdefender.com\\} 

	\and

	\IEEEauthorblockN{Dumitru Bogdan Prelipcean\\}
	\IEEEauthorblockA{"Al.I. Cuza" University - Faculty of Computer Science\\
	Bitdefender Cyber Threat Intelligence Lab\\
	Ia\c si, Rom\^ ania\\
	Email: bprelipcean@bitdefender.com\\} 

	

	

	
}

\maketitle
\begin{abstract}
The current threat landscape is diverse and has lately been shifting from the binary executable application to a more light-coded and data-oriented approach. Considering this, the use of Microsoft Office documents in attacks has increased. The number of malicious samples is high and the complexity of evasion techniques is also challenging. The VBA macros are highly used in enterprise environments with benign purposes, so, in terms of detection, the number of false alarms should be close to zero. 
\par In this paper we discuss and propose a solution which focuses on keeping the rate of false positives as low as possible and, at the same time, maximizes the detection rate. 
\end{abstract}

\begin{IEEEkeywords}
	Malware detection, classifier, feature extraction, feature selection, VBA macro, malicious macros, Office documents
\end{IEEEkeywords}
\section{Introduction}
Introduction
\par

\section{Related work}
The problem of detecting malicious macros inside Office documents has determined the creation of multiple analysis tools meant to help researchers and, in recent years, multiple studies were conducted to test new approaches to detect malicious files. The main focus of these approaches are the structural properties of the macro code that derive from obfuscation or general features that try to describe general aspects of the code and behavioral features. 
\par
In addition to having analysis capabilities, the mentioned tools are also aimed at detecting malicious traits inside Office files. OfficeMalScanner is a free forensic tool that can be used to scan for malicious traces such as shellcode heuristics, PE-files or embedded OLE streams in legacy files which use the old OLE binary format. Another tool aimed at detecting vulnerabilities is OfficeCAT which is an Office file checker based searching specific signatures to determine if a legacy file is unsafe. Microsoft OffVis is a tool that helps with the visualisation and understanding of Office files in the OLE format (.doc, .xls, .ppt) and can also identify exploits by checking vulnerability signatures. pyOLEScanner is a python script inspired by OfficeMalScanner which has the ability to scan and evaluate a file in order to determine if it could be malicious.
\par
A machine learning approach to malicious macro detection was presented by Sangwoo Kim et al. It is based on 15 discriminant static features that describe certain obfuscation techniques used in the code  These features are associated with four types of obfuscation methods: random obfuscation, split obfuscation, encoding obfuscation and logic obfuscation. The training was performed on a set of 2,537 files (773 benign, 1,764 malicious) and four classifiers were tested: SVM (95,5\%), Random Forest (96,5\%), Multi-Layer Perceptron (97\%), Linear Discriminant Analysis (90,1\%), Bernoulli Naive Bayes (89,1\%). 
\par
A similar approach to macro detection is described by Ed Aboud and Darragh O’Brien. They are using features based on the macro found inside Office files, but these features can describe more generic properties of the VBA code, such as Macro Keywords, Count of Integer Variables, etc. Additionally, an OCR library is used to identify certain phrases used in social engineering attack such as Enable Content and Previous Version. The images which contain these phrases are extracted by searching image magic numbers in the OLE binary. Using the extracted features, they created a feature vector which was the input for five different classifiers. The training was performed on a set of 400 samples (200 benign, 200 malicious) and the testing phase consisted of 528 malicious samples and 83 benign samples. The performance of these classifiers was evaluated in terms of the TPR: Random Forest (98,9875\%), KNeighbors (97,527\%), DecisionTree (98,225\%), GaussianNB (97,02\%).
In “Macro Malware Detection using Machine Learning Techniques”, Sergio De los Santos and Jose Torres propose using the features extracted by the python libraries OleFile and OleVBA. They establish a feature vector composed of 45 features, but they highlight the fact that these features are not distinct, some being obtained by discretizing features that are not boolean in nature. Their system was trained  uses the following classifiers with the following accuracy values during test phase: SVM (89\%),  Decision Tree (95\%), Random Forest (94\%) and Neural Networks (99\%). 
\par
ALDOCX is a framework implemented by Nisam et al. based on machine learning classifiers which uses features related to the structure of the paths in a ZIP archive, a methodology named SFEM. The extracted properties serve as input for an SVM classifier paired with an Active Learning component which allows the system to continuously update the detection model. The classifier was trained on a set of 16,811 samples (327 malicious, 16,484 benign) and achieved a TPR of 93.34\%, FPR of 0.19\% and 99.67\% accuracy. It is important to point out that, due to the process of choosing the features (they are paths inside a zip file), this framework only supports the newer OOXML file format (.docx, .xlsx).
\par
Although not a detection method itself, Microsoft implemented back in 2015 the Antimalware Scan Interface (AMSI) which allows applications on Windows 10 to request a scan of the memory buffer by an AV solution. This allows the logging of macro behaviour at runtime which can be used to analyze macro code in it’s deobfuscated state.
\par
As we can observe from the research work and tools mention above, most of the detection techniques are based around obfuscation, dynamic analysis, static, signature-based evaluation or a rather low number of features extracted from the macro code inside the VBA project found in Office files.
In the following sections, we will present our solution regarding the detection of malicious macro code using features which can describe the sample space in a more generic manner.
\section{Problem description}

The key points of the VBA-based malicious software are the accessibility and the popularity of Office Products. This is due to the fact that those products offer a scripting component named \textit{macro}. A \textit{macro} is a scripting system initially aimed at automating certain tasks. Being written in Visual Basic, it has access to the majority of the functions implemented in Windows API. This, along with the mass usage of Office products, both in business and non-business fields, makes it one of the favourite tools for malware creators. According to a statistic made by Spiceworks\footnote{https://community.spiceworks.com/software/articles/2873-data-snapshot-the-state-of-productivity-suites-in-the-workplace}, nearly 82\% of companies use programs from the Microsoft Office package. Knowing this, it is easier for an attacker to build a malicious program which will be granted to run, with a certain probability, on the target machine. However, over the past years, Microsoft implemented different security patches which fixed known vulnerabilities and nowadays the execution of a macro without user consent is denied. Because of these measures, the attackers had to implement social engineering methods to trick users into executing malicious macros.
\par
Malicious macros are used both as a delivery method and as a standalone malicious entity. If, in the past, the main target of a malicious document was to spread itself across the network, nowadays the potential offered by those products made the malware creators to rethink their usage. It is more common to see a document whose target is to steal sensitive information from companies or to deliver a destructive component, for example, a ransomware. According to CISCO\footnote{https://www.cisco.com/c/dam/m/digital/elq-cmcglobal/witb/acr2018/acr2018final.pdf}, more than 38\% of malicious file extensions found in 2017 in groups of malicious prevalent files were Microsoft Office formats.
\par
Considering the above observation, we developed a solution to detect malicious macros inside an office file and keep the rate of false positives as low as possible, all with respect to performance requirements. Our system is based on a derived version of the perceptron, OSC\cite{OSC}, and it is used over a set of features extracted from the binary Visual Basic Application, which is found inside an office document. The system extracts the macros within a file, applies a lexical processing, extracts the sequence of features and feeds them to the machine learning algorithm. In this way, we provide a complete system to detect if an Office file is indeed malicious or not.
\par
Moreover, taking into account that the OSC is keeping the rate of false positives for its models at a small percentage, we can easily use and integrate them into an AV solution. Having a small amount of false positives is mandatory due to the fact that Office products are used in critical business infrastructures.

\section{Database and features}
\par
The classifier described in this paper was tested on a set of samples provided by the Bitdefender Cyber Threat Intelligence Laboratory. The initial dataset consisted of 217,557 Office files and VBA projects (.doc, .docx, .xls, .xlsx) collected from June 2017 to May 2019, 176,314 malicious and 41,243 benign. In the further presentation of the features and samples selection we will use the word “token” to describe an atomic element of the VBA language, such as a keyword, a variable name, a constant value or an operator. The samples were passed through a filter which removed documents whose macros had 32 tokens or less,  because we considered that those files do not provide relevant information for our training process. 
\par
As a result of the filtering process, the number of malicious samples was unchanged and the number of benign samples was reduced to 38,668. For the training process, we selected 30,000 of the 38,668 benign samples and 10,000 malicious samples in a random fashion.
\par
The features used in the training process have been extracted by various processing methods applied to the tokens found in the macro code of the Office files. This static approach of the features extraction is motivated by the need of building a fast evaluation method. A dynamic feature extraction procedure would imply the execution of a sample in a controlled environment in order to observe its behaviour, which is very time consuming. In a real world scenario, the nature of a document needs to be established as fast as possible. Furthermore, the characteristics observed by executing a sample can be extracted by processing the macro code.  In order to create a model that is as generic as possible, we extracted properties that are specific to clean samples, to malicious ones and properties which can describe both classes of samples. By analyzing samples from both categories and developping processing methods we extracted a set of 536 features which describe general characteristics of the macro code (for example, the number of variables containing digits in their name), obfuscation features (for example, the number of statements in an assignment) and also behavioral characteristics, such as a function call with suspicious parameters.
\par
Due to the aforementioned performance restrictions, we will use only boolean features. This method also reduces the space needed to store the values of the features. Because some these values are not inherently boolean, the features need to be discretized. The discretization process of a feature involves creating a set of intervals which enclose its values. The inclusion of a value within one of these intervals will constitute a new feature. The process used for discretizing the values of a feature aimed to create intervals which include elements whose frequency and value are very close.  For example, we observed that some malicious samples have a higher number of small functions than benign ones. This information led us to the selection of 12 intervals which represent the values of the feature in a way that provides more relevant data. (Table I)
\begin{table}[ht]
    \centering
    \begin{tabular}{| c | c | c | c | c | c | c | }
    \hline
    Name & Interval & Feature Name\\ \hline
    NR SMALL FUNCTIONS & $[1, 3)$ & \textit{"under-3-functions"}  \\ \hline
    NR SMALL FUNCTIONS & $[3, 4)$ & \textit{"under-4-functions"}  \\ \hline
    ... & ... & ... \\ \hline   
    NR SMALL FUNCTIONS & $[953, 954) $ & \textit{"under-954-methods"}  \\ \hline
    \end{tabular}
    \caption{Example of discretization} 
    \label{tab:discretizationeg}
\end{table}
\par
By discretizing the values, from an initial set of 536 features, we arrived at a feature vector of 2977 properties. By design, our approach will use only 256 features. These features are chosen using a process named Conditional Mutual Information Maximization. The goal of this process is to select a subset of features which carries as much information as possible while minimizing the information shared between these features in order to eliminate redundant information.
\section{Results}
\par
As we previously stated, we built our detection model around the Perceptron algorithm and our objective is to construct the model such that it represents a generic description of the chosen dataset.
\par
Before we present the results, it is important to state that one of the main focuses of the training and testing phases is to achieve a very low false positives rate. This is especially important in the context of an AV solution because the detection of a benign sample can have serious consequences on a user's working environment. Consequently, we chose to use a modified version of the Perceptron algorithm name OSC, which incorporates an extra stage in the training algorithm to insure that all the samples marked as benign are correctly classified. Furthermore, the evaluation of a sample in the OSC algorithm is identical to the one used in the Perceptron algorithm, which is an advantage from the standpoint of performance.
\par
In order to train our model we used two approaches, $OSC$ and $OSC_{U}$, with two slightly different discretization methods. The first approach involves using almost all our records. It is important to express the fact that our training data does not involve any inconsistencies. Using the first discretization method, we trained our model ({$OSC_{D1}$}) for 500 iterations and we developed an accuracy of 99.68\% and sensitivity of 98.73\%. 
\par
The second approach, named $OSC_{U}$,  involved removing duplicate records from the entries used as inputs for the training algorithm. In this regard, if we encountered multiple records with the same feature vector (and labeled the same) we only kept one record.
\par
As a result of this filtering, the number of training samples was reduced to 14,495 (11,636 benign, 2859 malicious). This new model ({$OSC_{UD1}$}) yielded an accuracy of 99.65\% and sensitivity of 98.25\%.
\par
We have also experimented with a slightly different discretization method which has the effect of increasing the length of the intervals associated with the features. Using this new method, we trained another two models ({$OSC_{D2}$} and {$OSC_{UD2}$} for 500 iterations. The ({$OSC_{D2}$} model has developed an accuracy of 97.72\% and sensitivity of 98.91\% and
the{$OSC_{UD2}$} had an accuracy of 99.69\% and sensitivity of 98.44\%. We specify that the filtering process used for {$OSC_{UD2}$} reduced the training dataset to 14255 samples, 11,493 benign and 2762 malicious.
\begin{table}[ht]
    \centering
    \begin{tabular}{| c | c | c | c | c | c | c | }
    \hline
    Name & Sensitivity & Accuracy\\ \hline
    \textit{$OSC_{D1}$} --- 500 & $98.73\%$ &  99.68\% \\ \hline
    \textit{$OSC_{UD1}$} --- 500 & $98.25\%$ &  99.65\% \\ \hline
    \textit{$OSC_{D2}$} --- 500  & $98.91\%$ &  99.72\% \\ \hline
    \textit{$OSC_{UD2}$} --- 500 & $98.44\%$ &  99.69\% \\ \hline
    \end{tabular}
    \caption{Comparative training results} 
    \label{tab:trainingresults}
\end{table}
\par
We constructed a test set of samples in order to compare our two proposed models. We selected 8668 benign samples and 10,000 malicious samples. The results of these tests can be observed in the following table:
\begin{table}[ht]
    \centering
    \begin{tabular}{| c | c | c | c | c | c | c | }
    \hline
    Name & FP & FP Rate & TP & Sensitivity\\ \hline
    \textit{$OSC_{D1}$} --- 500 & $23$ & 0.26\% & 9231 & 92.31\% \\ \hline
    \textit{$OSC_{UD1}$} --- 500 & $16$ &  0.18\% & 9194 & 91.94\% \\ \hline
    \textit{$OSC_{D2}$} --- 500 & $28$ & 0.32\% & 9299 & 92.99\% \\ \hline
    \textit{$OSC_{UD2}$} --- 500 & $9$ &  0.10\% & 9037 & 90.37\% \\ \hline
    \end{tabular}
    \caption{Results on test dataset} 
    \label{tab:testresults}
\end{table}
\par
Considering these results, we can state that there is no model that can be considered the best.  Although the models built using the second discretization method have a higher accuracy and sensitivity in the testing phase, they perform worse in terms of FP rate ({$OSC_{D2}$} has a higher FP rate than {$OSC_{D1}$}) or sensitivity ({$OSC_{UD2}$} has a lower sensitivity than {$OSC_{UD1}$}) on the test dataset. Taking into account these results, we can state that the choice of "the best model" means making a trade-off between the FP rate and the sensitivity and one must choose a model to best fit his needs. Due to the fact that one of our goals is to construct a model that has a low FP rate, the best model in our case is {$OSC_{UD2}$}.
\section{Conclusion}
\par
If we evaluate the current threat landscape, we observe that macro malware has become a prevalent security problem. 
Traditional “signature-based” detections have the advantage of being a safe detection method with low false positive rates, being easy to generate and maintain. Although effective on short-term, these methods do not produce a solution which has the capacity to generalize the malicious features of new threats and can be easy to bypass. Analyzing the results presented in this paper, we can state that our solution, along with existing protection measures is a good step toward a highly generic and accurate detection system. 
\par
We acknowledge the fact that our perceptron model is more difficult to maintain and train and requires a large collection of samples and compute power. Despite these requirements, this type of model is not as volatile and is capable of better generalizing the characteristics of different samples.
\par
In a real world scenario, we consider that a middle way should be chosen. The traditional methods provide stability to the detection process of known malware families, while the perceptron model can be more exploratory in nature and discover new unknown threats.
\par
Consequently, the solution presented in this paper is essential to a cutting edge security solution and an actual necessity in the current state of growing macro malware threats.


\bibliographystyle{IEEEtran}
\bibliography{paper}


\end{document}
