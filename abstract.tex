\begin{abstract}
In recent years, VBA macro malware has been growing in popularity and so has the necessity of detecting it. While originally developed as a powerful scripting language to help users automate tasks and create macro-driven applications, it became a prevalent infection vector due to it’s extended capabilities, such as accessing the native Windows API. As a response to the growing threat of malicious macros, Microsoft has implemented several security features to prevent the execution of unwanted code. Nowadays, as a response to these measures, the attackers distribute the malicious documents through spear-phishing emails and use social engineering techniques in order to persuade the victim to enable the execution of the malicious macros.
\par
This paper aims to explain the current state of detections regarding these types of malicious files and analyzes the possibility of improving existing macro detections via machine learning techniques. The new detection method is based around a derived version of the Perceptron algorithm which builds a detection model focused on the properties of the macro code extracted from the VBA project of Office files. The paper also explores the advantages of using features that are not based around a particularity of the malicious file.
\end{abstract}

\begin{IEEEkeywords}
	Machine learning, feature extraction, feature selection, VBA macro, malicous macros
\end{IEEEkeywords}